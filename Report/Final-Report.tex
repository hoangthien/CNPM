\documentclass{article}
\usepackage[T5]{fontenc}
\usepackage[utf8]{inputenc}
\usepackage{graphicx}

\title{Báo cáo Nhập môn Công nghệ phần mềm}
\date{20-12-2014}
\author{
	Nguyễn Việt Hùng - 12520160
	Đỗ Trung Hiếu - 12520135
	Vũ Thành Nhân - 12520302
}

\begin{document}
	\tableofcontents
	\newpage
	%\maketitle
	
	\section{Lời nói đầu}
	Đồ án môn học Nhập môn Công nghệ Phần mềm là phân tích và xây dựng ứng dụng quản lý sổ Tiết kiệm cho ngân hàng, với các chức năng chính bao gồm: tạo sổ mới, gửi tiền, rút tiền, tìm kiếm sổ, quản lý \& thay đổi các chính sách tiết kiệm, lập báo cáo thống kê.
	
	Chương trình nhằm giúp nhóm áp dụng các kiến thức về quy trình phát triển phần mềm, cụ thể là sử dụng mô hình phát triển phần mềm \textit{Thác nước - Waterfall} vào bài toán này.
	
	\section{Khảo sát hiện trạng và xác định yêu cầu}
	\subsection{Nhu cầu thực tế và hiện trạng}
	
	Trong 10 năm trở lại đây (2004-2014) hệ thống các ngân hàng đã tiến
	hành công nghệ thông tin vào hầu hết các nghiệp vụ của ngân hàng, việc ứng dụng đó đã hỗ trợ hầu hết các nghiệp vụ một cách tự động trong ngân hàng và hỗ trợ rất nhiều trong công tác quản lý. Một trong số đó có việc \emph{\textbf{quản lý sổ tiết kiệm}} của khách hàng tuy nhiên mức độ ứng dụng Công nghệ thông tin của nước ta so với khu vực và thế giới vẫn còn lạc hậu. Điều này dẫn đến nhiều vấn đề khó khăn trong công tác hội nhập và mở rộng dịch vụ của ngân hàng.
	
	Đứng trước nhu cầu hội nhập và mở rộng hệ thống Ngân hàng mang tầm vóc khu vực và thế giới, đề tài \textbf{"\textit{quản lý sổ tiết kiệm}"} ra đời. Đề tài nhằm thử nghiệm chương trình quản lý nghiệp vụ quản lý sổ tiết kiệm ứng dụng công nghệ thông tin, làm tiền đề cho quá trình mở rộng và hội nhập với các dịch vụ ngân hàng hiện đại.
	
	
	\subsection{Yêu cầu chức năng}
	
		\begin{itemize}
			\item Theo yêu cầu của đồ án môn học, phần mềm quản lý sổ tiết kiệm cần đáp ứng các yêu cầu sau:
			\begin{itemize}
				\item Mở sổ tiết kiệm.
				\item Thực hiện giao dịch gửi tiền.
				\item Thực hiện giao dịch rút tiền.
				\item Tìm kiếm sổ tiết kiệm.
				\item Lập báo cáo thống kê.
				\item Thay đổi các chính sách tiết kiệm.
			\end{itemize}		
		\end{itemize}
		
		\begin{enumerate}
			\item Mở sổ tiết kiệm
			\item Thực hiện giao dịch gửi tiền.
			\item Thực hiện giao dịch rút tiền.
			\item Tìm kiếm sổ tiết kiệm.
			\item Lập báo cáo thống kê.
			\item Thay đổi các chính sách tiết kiệm.
		\end{enumerate}
		
		
	
	\subsection{Yêu cầu phi chức năng}
	
	\section{Đặc tả yêu cầu}
	
	\section{Thiết kế giao diện}
	\subsection{Danh sách các màn hình}
	\subsection{Mô tả chi tiết các màn hình}
	
	\section{Thiết kế dữ liệu}
	\subsection{Sơ đồ logic}
	\subsection{Mô tả chi tiết các kiểu dữ liệu trong sơ đồ logic}
	
	\section{Mô hình và thiết kế xử lý}
	\subsection{Danh sách các hàm xử lý}
	\subsection{Các sơ đồ kết hợp}
	
	\section{Kế hoạch lập trình và kiểm thử đơn vị}
	
	\section{Kết luận}
	
	Trong quá trình thực hiện đồ án, nhóm còn vấp phải nhiều sai sót...
	
	
	Chia nhỏ mục:
	
	Nhận định tình hình nhóm làm được
	
	Còn thiếu gì?
	
	Bài học???
	
	Sau này...
	
\end{document}}